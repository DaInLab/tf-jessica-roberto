% Options for packages loaded elsewhere
\PassOptionsToPackage{unicode}{hyperref}
\PassOptionsToPackage{hyphens}{url}
%
\documentclass[
]{article}
\usepackage{amsmath,amssymb}
\usepackage{lmodern}
\usepackage{iftex}
\ifPDFTeX
  \usepackage[T1]{fontenc}
  \usepackage[utf8]{inputenc}
  \usepackage{textcomp} % provide euro and other symbols
\else % if luatex or xetex
  \usepackage{unicode-math}
  \defaultfontfeatures{Scale=MatchLowercase}
  \defaultfontfeatures[\rmfamily]{Ligatures=TeX,Scale=1}
\fi
% Use upquote if available, for straight quotes in verbatim environments
\IfFileExists{upquote.sty}{\usepackage{upquote}}{}
\IfFileExists{microtype.sty}{% use microtype if available
  \usepackage[]{microtype}
  \UseMicrotypeSet[protrusion]{basicmath} % disable protrusion for tt fonts
}{}
\makeatletter
\@ifundefined{KOMAClassName}{% if non-KOMA class
  \IfFileExists{parskip.sty}{%
    \usepackage{parskip}
  }{% else
    \setlength{\parindent}{0pt}
    \setlength{\parskip}{6pt plus 2pt minus 1pt}}
}{% if KOMA class
  \KOMAoptions{parskip=half}}
\makeatother
\usepackage{xcolor}
\IfFileExists{xurl.sty}{\usepackage{xurl}}{} % add URL line breaks if available
\IfFileExists{bookmark.sty}{\usepackage{bookmark}}{\usepackage{hyperref}}
\hypersetup{
  pdftitle={Trabalho Final - Casos de COVID-19 em Bauru},
  pdfauthor={Jessica Balbino Duarte},
  hidelinks,
  pdfcreator={LaTeX via pandoc}}
\urlstyle{same} % disable monospaced font for URLs
\usepackage[margin=1in]{geometry}
\usepackage{graphicx}
\makeatletter
\def\maxwidth{\ifdim\Gin@nat@width>\linewidth\linewidth\else\Gin@nat@width\fi}
\def\maxheight{\ifdim\Gin@nat@height>\textheight\textheight\else\Gin@nat@height\fi}
\makeatother
% Scale images if necessary, so that they will not overflow the page
% margins by default, and it is still possible to overwrite the defaults
% using explicit options in \includegraphics[width, height, ...]{}
\setkeys{Gin}{width=\maxwidth,height=\maxheight,keepaspectratio}
% Set default figure placement to htbp
\makeatletter
\def\fps@figure{htbp}
\makeatother
\setlength{\emergencystretch}{3em} % prevent overfull lines
\providecommand{\tightlist}{%
  \setlength{\itemsep}{0pt}\setlength{\parskip}{0pt}}
\setcounter{secnumdepth}{-\maxdimen} % remove section numbering
\ifLuaTeX
  \usepackage{selnolig}  % disable illegal ligatures
\fi

\title{Trabalho Final - Casos de COVID-19 em Bauru}
\author{Jessica Balbino Duarte}
\date{2022-03-14}

\begin{document}
\maketitle

\hypertarget{introduuxe7uxe3o}{%
\section{1. Introdução}\label{introduuxe7uxe3o}}

A Covid-19 é uma infecção respiratória aguda causada pelo coronavírus
SARS-CoV-2, potencialmente grave, de elevada transmissibilidade e de
distribuição global.

O SARS-CoV-2 é um betacoronavírus descoberto em amostras de lavado
broncoalveolar obtidas de pacientes com pneumonia de causa desconhecida
na cidade de Wuhan, província de Hubei, China, em dezembro de 2019.
Pertence ao subgênero Sarbecovírus da família Coronaviridae e é o sétimo
coronavírus conhecido a infectar seres humanos.

\hypertarget{objetivo}{%
\section{2. Objetivo}\label{objetivo}}

O objetivo deste relatório é proporcionar uma Análise Exploratória de
Dados sobre como a pandemia de Coronavirus (Covid-19) afeta a cidade de
Bauru, situada no interior do Estado de São Paulo, no Brasil.

Os dados coletados foram publicados no jornal local usando dados do
sistema municipal de saúde.

\begin{verbatim}
## Carregando pacotes exigidos: tidyverse
\end{verbatim}

\begin{verbatim}
## -- Attaching packages --------------------------------------- tidyverse 1.3.1 --
\end{verbatim}

\begin{verbatim}
## v ggplot2 3.3.5     v purrr   0.3.4
## v tibble  3.1.6     v dplyr   1.0.8
## v tidyr   1.2.0     v stringr 1.4.0
## v readr   2.1.2     v forcats 0.5.1
\end{verbatim}

\begin{verbatim}
## -- Conflicts ------------------------------------------ tidyverse_conflicts() --
## x dplyr::filter() masks stats::filter()
## x dplyr::lag()    masks stats::lag()
\end{verbatim}

\begin{verbatim}
## Carregando pacotes exigidos: readxl
\end{verbatim}

\begin{verbatim}
## Warning: package 'readxl' was built under R version 4.1.3
\end{verbatim}

\hypertarget{anuxe1lise-exploratuxf3ria-de-dados}{%
\section{3. Análise Exploratória de
Dados}\label{anuxe1lise-exploratuxf3ria-de-dados}}

No gráfico da Figura 1, temos Mortes por Covid-19 em Bauru (Distribuição
por idade).

Com em relação ao tipo de hospitalização, nota-se pelo gráfico da Figura
2 que o número de óbitos da pandemia são relativamente similares nos
dois tipos de hospitalização, ou seja, pública e privada.

\includegraphics{trabalho-final---covid-19_files/figure-latex/hospitalizacao_1-1.pdf}

Sobre a permanência hospitalar, a Figura 3 apresenta uma ampla faixa de
número de dias de permanência hospitalar daqueles que chegaram à óbito.

\begin{verbatim}
## Carregando pacotes exigidos: lubridate
\end{verbatim}

\begin{verbatim}
## 
## Attaching package: 'lubridate'
\end{verbatim}

\begin{verbatim}
## The following objects are masked from 'package:base':
## 
##     date, intersect, setdiff, union
\end{verbatim}

\includegraphics{trabalho-final---covid-19_files/figure-latex/permanencia-1.pdf}

Uma análise apresentada no gráfico da Figura 4 mostra que o número dos
óbitos sofreu um grande avanço nos períodos mais críticos da pandemia,
assim como no Brasil.

\includegraphics{trabalho-final---covid-19_files/figure-latex/periodo_variacao-1.pdf}

\end{document}
